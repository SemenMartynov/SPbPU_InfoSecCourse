\documentclass[a4paper, 12pt]{article}		% general format

%%%% Charset
\usepackage{cmap}							% make PDF files searchable and copyable
\usepackage[utf8x]{inputenc}				% accept different input encodings
\usepackage[T2A]{fontenc}					% russian font
\usepackage[russian]{babel}					% multilingual support (T2A)

%%%% Graphics
\usepackage[dvipsnames]{xcolor}			% driver-independent color extensions
\usepackage{graphicx}						% enhanced support for graphics
\usepackage{wrapfig}						% produces figures which text can flow around

%%%% Math
\usepackage{amsmath}						% American Mathematical Society (AMS) math facilities
\usepackage{amsfonts}						% fonts from the AMS
\usepackage{amssymb}						% additional math symbols

%%%% Typograpy (don't forget about cm-super)
\usepackage{microtype}						% subliminal refinements towards typographical perfection
\linespread{1.3}							% line spacing
\usepackage[left=2.5cm, right=1.5cm, top=2.5cm, bottom=2.5cm]{geometry}
\setlength{\parindent}{0pt}					% we don't want any paragraph indentation
\usepackage{parskip}						% add distance between paragraphs

%%%% Tables
\usepackage{tabularx}						% Normal tables
\usepackage{multirow}						% for tabular
\usepackage{hhline}							% for tabular


%%%% Other
\usepackage{url}							% verbatim with URL-sensitive line breaks
\usepackage{fancyvrb}						% verbatim with box
\setcounter{secnumdepth}{5}					%
%------------------------------------------------------------------------------
\usepackage{listings}						% typeset source code listings

% Цвета для кода
\definecolor{string}{HTML}{101AF9}			% цвет строк в коде
\definecolor{comment}{HTML}{3F7F5F}		% цвет комментариев в коде
\definecolor{keyword}{HTML}{5F1441}		% цвет ключевых слов в коде
\definecolor{morecomment}{HTML}{8000FF}	% цвет include и других элементов в коде
\definecolor{captiontext}{HTML}{FFFFFF}	% цвет текста заголовка в коде
\definecolor{captionbk}{HTML}{999999}		% цвет фона заголовка в коде
\definecolor{bk}{HTML}{FFFFFF}				% цвет фона в коде
\definecolor{frame}{HTML}{999999}			% цвет рамки в коде

% Настройки отображения кода
\lstset{
	language=C++,							% Язык кода по умолчанию
	morekeywords={*,...},					% если хотите добавить ключевые слова, то добавляйте
	% Цвета
	keywordstyle=\color{keyword}\ttfamily\bfseries,
	stringstyle=\color{string}\ttfamily,
	commentstyle=\color{comment}\ttfamily\itshape,
	morecomment=[l][\color{morecomment}]{\#},
	% Настройки отображения
	breaklines=true,						% Перенос длинных строк
	basicstyle=\ttfamily\footnotesize,		% Шрифт для отображения кода
	backgroundcolor=\color{bk},				% Цвет фона кода
	%frame=lrb,xleftmargin=\fboxsep,xrightmargin=-\fboxsep, % Рамка, подогнанная к заголовку
	frame=tblr								% draw a frame at all sides of the code block
	rulecolor=\color{frame},				% Цвет рамки
	tabsize=2,								% tab space width
	showstringspaces=false,					% don't mark spaces in strings
	% Настройка отображения номеров строк. Если не нужно, то удалите весь блок
	numbers=left,							% Слева отображаются номера строк
	stepnumber=1,							% Каждую строку нумеровать
	numbersep=5pt,							% Отступ от кода
	numberstyle=\small\color{black},		% Стиль написания номеров строк
	% Для отображения русского языка
	extendedchars=true,
	literate={Ö}{{\"O}}1
	 	{Ä}{{\"A}}1
	 	{Ü}{{\"U}}1
		{ß}{{\ss}}1
		{ü}{{\"u}}1
		{ä}{{\"a}}1
		{ö}{{\"o}}1
		{~}{{\textasciitilde}}1
		{а}{{\selectfont\char224}}1
		{б}{{\selectfont\char225}}1
		{в}{{\selectfont\char226}}1
		{г}{{\selectfont\char227}}1
		{д}{{\selectfont\char228}}1
		{е}{{\selectfont\char229}}1
		{ё}{{\"e}}1
		{ж}{{\selectfont\char230}}1
		{з}{{\selectfont\char231}}1
		{и}{{\selectfont\char232}}1
		{й}{{\selectfont\char233}}1
		{к}{{\selectfont\char234}}1
		{л}{{\selectfont\char235}}1
		{м}{{\selectfont\char236}}1
		{н}{{\selectfont\char237}}1
		{о}{{\selectfont\char238}}1
		{п}{{\selectfont\char239}}1
		{р}{{\selectfont\char240}}1
		{с}{{\selectfont\char241}}1
		{т}{{\selectfont\char242}}1
		{у}{{\selectfont\char243}}1
		{ф}{{\selectfont\char244}}1
		{х}{{\selectfont\char245}}1
		{ц}{{\selectfont\char246}}1
		{ч}{{\selectfont\char247}}1
		{ш}{{\selectfont\char248}}1
		{щ}{{\selectfont\char249}}1
		{ъ}{{\selectfont\char250}}1
		{ы}{{\selectfont\char251}}1
		{ь}{{\selectfont\char252}}1
		{э}{{\selectfont\char253}}1
		{ю}{{\selectfont\char254}}1
		{я}{{\selectfont\char255}}1
		{А}{{\selectfont\char192}}1
		{Б}{{\selectfont\char193}}1
		{В}{{\selectfont\char194}}1
		{Г}{{\selectfont\char195}}1
		{Д}{{\selectfont\char196}}1
		{Е}{{\selectfont\char197}}1
		{Ё}{{\"E}}1
		{Ж}{{\selectfont\char198}}1
		{З}{{\selectfont\char199}}1
		{И}{{\selectfont\char200}}1
		{Й}{{\selectfont\char201}}1
		{К}{{\selectfont\char202}}1
		{Л}{{\selectfont\char203}}1
		{М}{{\selectfont\char204}}1
		{Н}{{\selectfont\char205}}1
		{О}{{\selectfont\char206}}1
		{П}{{\selectfont\char207}}1
		{Р}{{\selectfont\char208}}1
		{С}{{\selectfont\char209}}1
		{Т}{{\selectfont\char210}}1
		{У}{{\selectfont\char211}}1
		{Ф}{{\selectfont\char212}}1
		{Х}{{\selectfont\char213}}1
		{Ц}{{\selectfont\char214}}1
		{Ч}{{\selectfont\char215}}1
		{Ш}{{\selectfont\char216}}1
		{Щ}{{\selectfont\char217}}1
		{Ъ}{{\selectfont\char218}}1
		{Ы}{{\selectfont\char219}}1
		{Ь}{{\selectfont\char220}}1
		{Э}{{\selectfont\char221}}1
		{Ю}{{\selectfont\char222}}1
		{Я}{{\selectfont\char223}}1
		{і}{{\selectfont\char105}}1
		{ї}{{\selectfont\char168}}1
		{є}{{\selectfont\char185}}1
		{ґ}{{\selectfont\char160}}1
		{І}{{\selectfont\char73}}1
		{Ї}{{\selectfont\char136}}1
		{Є}{{\selectfont\char153}}1
		{Ґ}{{\selectfont\char128}}1
}

% Для настройки заголовка кода
\usepackage{caption}
\DeclareCaptionFont{white}{\color{сaptiontext}}
\DeclareCaptionFormat{listing}{\parbox{\linewidth}{\colorbox{сaptionbk}{\parbox{\linewidth}{#1#2#3}}\vskip-4pt}}
%\captionsetup[lstlisting]{format=listing,labelfont=white,textfont=white}
\renewcommand{\lstlistingname}{Листинг} % Переименование Listings в нужное именование структуры

%------------------------------------------------------------------------------
\author{Семён Мартынов\\<semen.martynov@gmail.com>}
\title{Отчет по лабораторной работе 2:\\Nmap Metasploit}
\begin{document}
\maketitle
\tableofcontents{}

%------------------------------------------------------------------------------
\newpage
\section{Утилита для исследования сети и сканер портов Nmap}

\subsection{Цель работы}

Изучение работы программы Nmap на примере локальной домашней сети и сети из виртуальных машин с Kali Linux и Metasploitable2.

\subsection{Ход работы}

Эта часть работы выполняется в домашней сети 192.168.124.0/24, построенной на технологиях Fast Ethernet (IEEE 802.3u) и WiFi (IEEE 802.11n).

\subsubsection{Опредление набора и версии сервисов запущенных на компьютере в диапазоне адресов}

\paragraph{Провести поиск активных хостов} Для сканирования сети буедет использована команда:
\begin{Verbatim}[frame=single]
nmap -sn 192.168.124.3-255
\end{Verbatim}

Сочетание ключей s и n приводит к быстрому сканированию (т.е. буз сканирования портов). Иногда это называют "ping scan" (и в старых версиях для этого испольховалось "-sP"). Цель задана диапазоном IP адресов, из которого исключен роутер, и машины, с которой проводилось сканирование.

Результат сканирования:
\begin{Verbatim}[frame=single]
$ nmap -sn 192.168.124.3-255

Starting Nmap 6.40 ( http://nmap.org ) at 2015-05-18 01:01 MSK
Nmap scan report for 192.168.124.4
Host is up (0.020s latency).
Nmap scan report for 192.168.124.100
Host is up (0.00030s latency).
Nmap scan report for 192.168.124.195
Host is up (0.034s latency).
Nmap scan report for 192.168.124.239
Host is up (0.042s latency).
Nmap scan report for 192.168.124.249
Host is up (0.038s latency).
Nmap done: 253 IP addresses (5 hosts up) scanned in 2.43 seconds
\end{Verbatim}

\paragraph{Определить открытые порты}

Опеределим состояние 10 наиболее популярных портах на хостах из того же диапазона (стоит отметить, что хостов стало меньше)

\begin{Verbatim}[frame=single]
$ nmap --top-ports 10 192.168.124.3-255

Starting Nmap 6.40 ( http://nmap.org ) at 2015-05-18 01:26 MSK
Nmap scan report for 192.168.124.4
Host is up (0.0057s latency).
PORT     STATE  SERVICE
21/tcp   closed ftp
22/tcp   open   ssh
23/tcp   closed telnet
25/tcp   closed smtp
80/tcp   closed http
110/tcp  closed pop3
139/tcp  closed netbios-ssn
443/tcp  closed https
445/tcp  closed microsoft-ds
3389/tcp closed ms-wbt-server

Nmap scan report for 192.168.124.100
Host is up (0.00026s latency).
PORT     STATE  SERVICE
21/tcp   closed ftp
22/tcp   open   ssh
23/tcp   closed telnet
25/tcp   closed smtp
80/tcp   closed http
110/tcp  closed pop3
139/tcp  open   netbios-ssn
443/tcp  closed https
445/tcp  open   microsoft-ds
3389/tcp closed ms-wbt-server

Nmap scan report for 192.168.124.244
Host is up (0.017s latency).
PORT     STATE  SERVICE
21/tcp   closed ftp
22/tcp   closed ssh
23/tcp   closed telnet
25/tcp   closed smtp
80/tcp   closed http
110/tcp  closed pop3
139/tcp  closed netbios-ssn
443/tcp  closed https
445/tcp  closed microsoft-ds
3389/tcp closed ms-wbt-server

Nmap scan report for 192.168.124.249
Host is up (0.035s latency).
PORT     STATE  SERVICE
21/tcp   closed ftp
22/tcp   closed ssh
23/tcp   closed telnet
25/tcp   closed smtp
80/tcp   closed http
110/tcp  closed pop3
139/tcp  closed netbios-ssn
443/tcp  closed https
445/tcp  closed microsoft-ds
3389/tcp closed ms-wbt-server

Nmap done: 253 IP addresses (4 hosts up) scanned in 2.79 seconds
\end{Verbatim}

\paragraph{Определить версии сервисов} Дополнение команды ключем V приведеёт к опредедлению версий (где это возможно).

\begin{Verbatim}[frame=single]
$ nmap -sV 192.168.124.3-255

Starting Nmap 6.40 ( http://nmap.org ) at 2015-05-18 01:34 MSK
Nmap scan report for 192.168.124.4
Host is up (0.029s latency).
Not shown: 999 closed ports
PORT   STATE SERVICE VERSION
22/tcp open  ssh     OpenSSH 6.8 (protocol 2.0)

Nmap scan report for 192.168.124.100
Host is up (0.00018s latency).
Not shown: 997 closed ports
PORT    STATE SERVICE     VERSION
22/tcp  open  ssh         (protocol 2.0)
139/tcp open  netbios-ssn Samba smbd 3.X (workgroup: IDEAPAD)
445/tcp open  netbios-ssn Samba smbd 3.X (workgroup: IDEAPAD)
1 service unrecognized despite returning data. If you know the
service/version, please submit the following fingerprint at
http://www.insecure.org/cgi-bin/servicefp-submit.cgi :
SF-Port22-TCP:V=6.40%I=7%D=5/18%Time=55591797%P=x86_64-pc-linux-gnu%r(NULL
SF:,29,"SSH-2\.0-OpenSSH_6\.6\.1p1\x20Ubuntu-2ubuntu2\r\n");

Nmap scan report for 192.168.124.244
Host is up (0.0080s latency).
All 1000 scanned ports on 192.168.124.244 are closed

Nmap scan report for 192.168.124.249
Host is up (0.0041s latency).
Not shown: 999 closed ports
PORT      STATE SERVICE    VERSION
62078/tcp open  tcpwrapped

Service detection performed. Please report any incorrect results at http://nmap.org/submit/ .
Nmap done: 253 IP addresses (4 hosts up) scanned in 54.77 seconds
\end{Verbatim}

\paragraph{Изучить файлы nmap-services, nmap-os-db, nmap-service-probes}

Служебный файл \textbf{nmap-services} представляет из себя базу данных портов и протоколов. Каждая запись имеет число, определяющее вероятность того, что порт может быть открыт.

Большинство строк имеют комментарии, которые Nmap игорирует, но пользователь может использовать GREP для получения информации о том или ином номере порта. Этот файл был изначально построен на базе список IANA, в котором сервисам распределялись порты (\url{http://www.iana.org/assignments/port-numbers}), но IANA не отслеживает порты троянов, червей и т.п., что является важным для многих пользователей Nmap.

Отрывок файла приведён ниже.

\begin{Verbatim}[frame=single]
qotd         17/tcp    0.002346  # Quote of the Day
qotd         17/udp    0.009209  # Quote of the Day
msp          18/udp    0.000610  # Message Send Protocol
chargen      19/tcp    0.002559  # ttytst source Character Generator
chargen      19/udp    0.015865  # ttytst source Character Generator
ftp-data     20/tcp    0.001079  # File Transfer [Default Data]
ftp-data     20/udp    0.001878  # File Transfer [Default Data]
ftp          21/tcp    0.197667  # File Transfer [Control]
ftp          21/udp    0.004844  # File Transfer [Control]
ssh          22/tcp    0.182286  # Secure Shell Login
ssh          22/udp    0.003905  # Secure Shell Login
telnet       23/tcp    0.221265
telnet       23/udp    0.006211
priv-mail    24/tcp    0.001154  # any private mail system
priv-mail    24/udp    0.000329  # any private mail system
smtp         25/tcp    0.131314  # Simple Mail Transfer
smtp         25/udp    0.001285  # Simple Mail Transfer
\end{Verbatim}

Файл \textbf{nmap-os-db} содержит сотни примеров того, как различные операционные системы видут себя в различных ситуациях, создаваемых Nmap. Этот файл разделен на блоки, известные как отпечатки пальцев (fingerprints) и с каждым отпечатком соотносится имя операционной системы и её общая классификация

\begin{Verbatim}[frame=single]
Fingerprint FreeBSD 7.0
Class FreeBSD | FreeBSD | 7.X | general purpose
SEQ(SP=100-10A%GCD=1-6%ISR=108-112%TI=I%II=I%SS=S%TS=21|22)
OPS(O1=M5B4NW8NNT11%O2=M578NW8NNT11%O3=M280NW8NNT11%O4=M5B4NW8NNT11%O5=M218NW8NNT11%O6=M109NNT11)
WIN(W1=FFFF%W2=FFFF%W3=FFFF%W4=FFFF%W5=FFFF%W6=FFFF)
ECN(R=Y%DF=Y%T=3B-45%TG=40%W=FFFF%O=M5B4NW8%CC=N%Q=)
T1(R=Y%DF=Y%T=3B-45%TG=40%S=O%A=S+%F=AS%RD=0%Q=)
T2(R=N)
T3(R=Y%DF=Y%T=3B-45%TG=40%W=FFFF%S=O%A=S+%F=AS%O=M109NW8NNT11%RD=0%Q=)
T4(R=Y%DF=Y%T=3B-45%TG=40%W=0%S=A%A=Z%F=R%O=%RD=0%Q=)
T5(R=Y%DF=Y%T=3B-45%TG=40%W=0%S=Z%A=S+%F=AR%O=%RD=0%Q=)
T6(R=Y%DF=Y%T=3B-45%TG=40%W=0%S=A%A=Z%F=R%O=%RD=0%Q=)
T7(R=Y%DF=Y%T=3B-45%TG=40%W=0%S=Z%A=S%F=AR%O=%RD=0%Q=)
U1(DF=N%T=3B-45%TG=40%IPL=38%UN=0%RIPL=G%RID=G%RIPCK=G%RUCK=G%RUD=G)
IE(DFI=S%T=3B-45%TG=40%CD=S)

Fingerprint Linux 2.6.17 - 2.6.24
Class Linux | Linux | 2.6.X | general purpose
SEQ(SP=A5-D5%GCD=1-6%ISR=A7-D7%TI=Z%II=I%TS=U)
OPS(O1=M400C%O2=M400C%O3=M400C%O4=M400C%O5=M400C%O6=M400C)
WIN(W1=8018%W2=8018%W3=8018%W4=8018%W5=8018%W6=8018)
ECN(R=Y%DF=Y%T=3B-45%TG=40%W=8018%O=M400C%CC=N%Q=)
T1(R=Y%DF=Y%T=3B-45%TG=40%S=O%A=S+%F=AS%RD=0%Q=)
T2(R=N)
T3(R=Y%DF=Y%T=3B-45%TG=40%W=8018%S=O%A=S+%F=AS%O=M400C%RD=0%Q=)
T4(R=Y%DF=Y%T=3B-45%TG=40%W=0%S=A%A=Z%F=R%O=%RD=0%Q=)
T5(R=Y%DF=Y%T=3B-45%TG=40%W=0%S=Z%A=S+%F=AR%O=%RD=0%Q=)
T6(R=Y%DF=Y%T=3B-45%TG=40%W=0%S=A%A=Z%F=R%O=%RD=0%Q=)
T7(R=Y%DF=Y%T=3B-45%TG=40%W=0%S=Z%A=S+%F=AR%O=%RD=0%Q=)
U1(DF=N%T=3B-45%TG=40%IPL=164%UN=0%RIPL=G%RID=G%RIPCK=G%RUCK=G%RUD=G)
IE(DFI=N%T=3B-45%TG=40%CD=S)
\end{Verbatim}

Файл \textbf{nmap-service-probes} содержит описания различных ситуация и ответного поведения сервиса. Это необходимо чтобы определить, какая программа прослушивает порт.

\begin{Verbatim}[frame=single]
##############################NEXT PROBE##############################
# DNS Server status request: http://www.rfc-editor.org/rfc/rfc1035.txt
Probe UDP DNSStatusRequest q|\0\0\x10\0\0\0\0\0\0\0\0\0|
ports 53,135
match domain m|^\0\0\x90\x04\0\0\0\0\0\0\0\0|
# This one below came from 2 tested Windows XP boxes
match msrpc m|^\x04\x06\0\0\x10\0\0\0\0\0\0\0|
[...]
##############################NEXT PROBE##############################
Probe UDP Help q|help\r\n\r\n|
ports 7,13,37
match chargen m|@ABCDEFGHIJKLMNOPQRSTUVWXYZ|
match echo m|^help\r\n\r\n$|
match time m|^[\xc0-\xc5]...$|
\end{Verbatim}

\paragraph{Добавить новую сигнатуру службы в файл nmap-service-probes} (для этого создать минимальный tcp server, добиться, чтобы при сканировании nmap указывал для него название и версию)

Исходный код простого TCP-сервера приведён в листинге 1.

\lstinputlisting[language=C++, caption={Пример простого TCP-сервера}]{../../Programming/main.c}

Простой запуск этого сервера можно обнаружить при помощи Nmap, но Nmap пока не знает, с чем имеет дело.

\begin{Verbatim}[frame=single]
$ nmap -sV -p 2404 192.168.124.4

Starting Nmap 6.40 ( http://nmap.org ) at 2015-05-18 03:42 MSK
Nmap scan report for 192.168.124.4
Host is up (0.0038s latency).
PORT     STATE SERVICE VERSION
2404/tcp open  echo

Service detection performed.
Please report any incorrect results at http://nmap.org/submit/ .
Nmap done: 1 IP address (1 host up) scanned in 41.28 seconds
\end{Verbatim}

Надо отметить, что основаня идея определена верно - это действительно эхо-севис. Но никаких данных о версии у нас нет. Теперь добавим описание сервиса в файл nmap-service-probes
\begin{Verbatim}[frame=single]
$ tail -n 7 /usr/share/nmap/nmap-service-probes
##############################NEXT PROBE##############################
# Simple TSP server.
Probe TCP simple-tcp-server-ver q|version\r\n|
rarity 9
ports 2404

match stcps m|^1\.0\.0$| p/Simple TCP Server/ v/1.0.0-3/
\end{Verbatim}

И снова проведём сканирование

\begin{Verbatim}[frame=single]
$ nmap -sV -p 2404 192.168.124.4
Starting Nmap 6.40 ( http://nmap.org ) at 2015-05-18 03:44 MSK
Nmap scan report for 192.168.124.4
Host is up (0.0035s latency).
PORT     STATE SERVICE VERSION
2404/tcp open  stcps   Simple TCP Server 1.0.0-3

Service detection performed. 
Please report any incorrect results at http://nmap.org/submit/ .
Nmap done: 1 IP address (1 host up) scanned in 6.21 seconds
\end{Verbatim}

\paragraph{Сохранить вывод утилиты в формате xml}

Вызов команды имеет следующий вид
\begin{Verbatim}[frame=single]
nmap -sV -p 2404 -oX - scanme.nmap.org 192.168.124.4
\end{Verbatim}

Результат представляет собой XML-файл
\begin{Verbatim}[frame=single]
<?xml version="1.0"?>
<?xml-stylesheet href="file:///usr/bin/../share/nmap/nmap.xsl"
                                                            type="text/xsl"?>
<!-- Nmap 6.40 scan initiated Mon May 18 03:47:51 2015
                 as: nmap -sV -p 2404 -oX - scanme.nmap.org 192.168.124.4 -->
<nmaprun scanner="nmap" args="nmap -sV -p 2404 -oX - scanme.nmap.org
              192.168.124.4" start="1431910071" startstr="Mon May 18 03:47:51
              2015" version="6.40" xmloutputversion="1.04">
<scaninfo type="connect" protocol="tcp" numservices="1" services="2404"/>
<verbose level="0"/>
<debugging level="0"/>
<host starttime="1431910071" endtime="1431910079"><status state="up"
                                       reason="conn-refused" reason_ttl="0"/>
<address addr="192.168.124.4" addrtype="ipv4"/>
<hostnames>
</hostnames>
<ports><port protocol="tcp" portid="2404"><state state="open"
              reason="syn-ack" reason_ttl="0"/><service name="stcps" product=
              "Simple TCP Server" version="1.0.0-3" method="probed" conf="10"/>
              </port>
</ports>
<times srtt="4122" rttvar="2991" to="100000"/>
</host>
<runstats><finished time="1431910079" timestr="Mon May 18 03:47:59 2015"
               elapsed="7.40" summary="Nmap done at Mon May 18 03:47:59 2015;
               2 IP addresses (1 host up) scanned in 7.40 seconds"
               exit="success"/><hosts up="1" down="1" total="2"/>
</runstats>
</nmaprun>
\end{Verbatim}

\paragraph{Исследовать различные этапы и режимы работы nmap с использованием утилиты Wireshark}

На рисунке 1 показано сканирование порта 2404 (по совпадению, он имеет имя iec-104). Видно, что в пакете передаётся запрос "version". А на рисунке 2 опрос 10 наиболее популярных портов.

\begin{figure}[h!]
\centering
\includegraphics[scale=0.37]{res/pic01}
\caption{Определение сервиса на порт 2040}
\end{figure}

\begin{figure}[h!]
\centering
\includegraphics[scale=0.37]{res/pic02}
\caption{Опрос 10 наиболее популярных портов}
\end{figure}

\newpage
\subsubsection{Просканировать виртуальную машину Metasploitable2 используя db\_nmap из состава metasploit\-framework}

% applications > kali linux > system services > metasploit > start

% applications > kali linux > top 10 security tools > metasploit framework

% msfconsole

Стоит отметить, что Metasploitable2 достаточно прожорлив в плане ресурсов, особенно по части оперативной памяти. Это является результатом большого количества запущеных сервисов.

\begin{Verbatim}[frame=single]
msf > db_nmap -v -sV 192.168.124.211
[*] Nmap: Starting Nmap 6.47 ( http://nmap.org ) at 2015-05-18 21:05 UTC
[*] Nmap: NSE: Loaded 29 scripts for scanning.
[*] Nmap: Initiating ARP Ping Scan at 21:05
[*] Nmap: Scanning 192.168.124.211 [1 port]
[*] Nmap: Completed ARP Ping Scan at 21:05, 0.05s elapsed (1 total hosts)
[*] Nmap: Initiating Parallel DNS resolution of 1 host. at 21:05
[*] Nmap: Completed Parallel DNS resolution of 1 host. at 21:05, 0.01s
                                                                      elapsed
[*] Nmap: Initiating SYN Stealth Scan at 21:05
[*] Nmap: Scanning 192.168.124.211 [1000 ports]
[*] Nmap: Discovered open port 22/tcp on 192.168.124.211
[*] Nmap: Discovered open port 5900/tcp on 192.168.124.211
[*] Nmap: Discovered open port 80/tcp on 192.168.124.211
[*] Nmap: Discovered open port 53/tcp on 192.168.124.211
[*] Nmap: Discovered open port 21/tcp on 192.168.124.211
[*] Nmap: Discovered open port 3306/tcp on 192.168.124.211
[*] Nmap: Discovered open port 445/tcp on 192.168.124.211
[*] Nmap: Discovered open port 23/tcp on 192.168.124.211
[*] Nmap: Discovered open port 25/tcp on 192.168.124.211
[*] Nmap: Discovered open port 111/tcp on 192.168.124.211
[*] Nmap: Discovered open port 139/tcp on 192.168.124.211
[*] Nmap: Discovered open port 2049/tcp on 192.168.124.211
[*] Nmap: Discovered open port 512/tcp on 192.168.124.211
[*] Nmap: Discovered open port 8180/tcp on 192.168.124.211
[*] Nmap: Discovered open port 6000/tcp on 192.168.124.211
[*] Nmap: Discovered open port 5432/tcp on 192.168.124.211
[*] Nmap: Discovered open port 1524/tcp on 192.168.124.211
[*] Nmap: Discovered open port 1099/tcp on 192.168.124.211
[*] Nmap: Discovered open port 6667/tcp on 192.168.124.211
[*] Nmap: Discovered open port 514/tcp on 192.168.124.211
[*] Nmap: Discovered open port 2121/tcp on 192.168.124.211
[*] Nmap: Discovered open port 8009/tcp on 192.168.124.211
[*] Nmap: Discovered open port 513/tcp on 192.168.124.211
[*] Nmap: Completed SYN Stealth Scan at 21:05, 0.55s elapsed (1000 total
                                                                       ports)
[*] Nmap: Initiating Service scan at 21:05
[*] Nmap: Scanning 23 services on 192.168.124.211
[*] Nmap: Completed Service scan at 21:05, 11.76s elapsed (23 services on 1
                                                                        host)
[*] Nmap: NSE: Script scanning 192.168.124.211.
[*] Nmap: Initiating NSE at 21:05
[*] Nmap: Completed NSE at 21:05, 0.16s elapsed
[*] Nmap: Nmap scan report for 192.168.124.211
[*] Nmap: Host is up (0.00030s latency).
[*] Nmap: Not shown: 977 closed ports
[*] Nmap: PORT     STATE SERVICE     VERSION
[*] Nmap: 21/tcp   open  ftp         vsftpd 2.3.4
[*] Nmap: 22/tcp   open  ssh         OpenSSH 4.7p1 Debian 8ubuntu1 (protocol
                                                                         2.0)
[*] Nmap: 23/tcp   open  telnet      Linux telnetd
[*] Nmap: 25/tcp   open  smtp        Postfix smtpd
[*] Nmap: 53/tcp   open  domain      ISC BIND 9.4.2
[*] Nmap: 80/tcp   open  http        Apache httpd 2.2.8 ((Ubuntu) DAV/2)
[*] Nmap: 111/tcp  open  rpcbind     2 (RPC #100000)
[*] Nmap: 139/tcp  open  netbios-ssn Samba smbd 3.X (workgroup: WORKGROUP)
[*] Nmap: 445/tcp  open  netbios-ssn Samba smbd 3.X (workgroup: WORKGROUP)
[*] Nmap: 512/tcp  open  exec        netkit-rsh rexecd
[*] Nmap: 513/tcp  open  login
[*] Nmap: 514/tcp  open  tcpwrapped
[*] Nmap: 1099/tcp open  rmiregistry GNU Classpath grmiregistry
[*] Nmap: 1524/tcp open  shell       Metasploitable root shell
[*] Nmap: 2049/tcp open  nfs         2-4 (RPC #100003)
[*] Nmap: 2121/tcp open  ftp         ProFTPD 1.3.1
[*] Nmap: 3306/tcp open  mysql       MySQL 5.0.51a-3ubuntu5
[*] Nmap: 5432/tcp open  postgresql  PostgreSQL DB 8.3.0 - 8.3.7
[*] Nmap: 5900/tcp open  vnc         VNC (protocol 3.3)
[*] Nmap: 6000/tcp open  X11         (access denied)
[*] Nmap: 6667/tcp open  irc         Unreal ircd
[*] Nmap: 8009/tcp open  ajp13       Apache Jserv (Protocol v1.3)
[*] Nmap: 8180/tcp open  http        Apache Tomcat/Coyote JSP engine 1.1
[*] Nmap: MAC Address: 08:00:27:6E:3D:DB (Cadmus Computer Systems)
[*] Nmap: Service Info: Hosts:  metasploitable.localdomain, localhost, 
     irc.Metasploitable.LAN; OSs: Unix, Linux; CPE: cpe:/o:linux:linux_kernel
[*] Nmap: Read data files from: /usr/bin/../share/nmap
[*] Nmap: Service detection performed. Please report any incorrect results at
                                                    http://nmap.org/submit/ .
[*] Nmap: Nmap done: 1 IP address (1 host up) scanned in 14.14 seconds
[*] Nmap: Raw packets sent: 1001 (44.028KB) | Rcvd: 1001 (40.120KB)
msf > 
\end{Verbatim}

\subsubsection{Выбрать пять записей из файла nmap-service-probes и описать их работу Выбрать один скрипт из состава Nmap и описать его работу}

\subsection{Выводы}


%------------------------------------------------------------------------------
\newpage
\section{Инструмент тестов на проникновение Metasploit}

\subsection{Цель работы}

\subsection{Ход работы}

\subsubsection{Описать последовательность действий для получения доступа к косоли}

\paragraph{Подключиться к VNC-серверу, получить доступ к консоли}

\paragraph{олучить список директорий в общем доступе по протоколу SMB}

\paragraph{Получить консоль используя уязвимость в vsftpd}

\paragraph{Получить консоль используя уязвимость в irc}

\paragraph{Armitage Hail Mary}

\subsubsection{Изучить три файла с исходным кодом эксплойтов или служебных скриптов на ruby и описать, что в них происходит}

\subsection{Выводы}


\end{document}